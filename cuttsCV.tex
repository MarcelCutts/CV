%%%%%%%%%%%%%%%%%%%%%%%%%%%%%%%%%%%%%%%%%
% Friggeri Resume/CV
% XeLaTeX Template
% Version 1.0 (5/5/13)
%
% This template has been downloaded from:
% http://www.LaTeXTemplates.com
%
% Original author:
% Adrien Friggeri (adrien@friggeri.net)
% https://github.com/afriggeri/CV
%
% License:
% CC BY-NC-SA 3.0 (http://creativecommons.org/licenses/by-nc-sa/3.0/)
%
% Important notes:
% This template needs to be compiled with XeLaTeX and the bibliography, if used,
% needs to be compiled with biber rather than bibtex.
%
%%%%%%%%%%%%%%%%%%%%%%%%%%%%%%%%%%%%%%%%%

\documentclass[]{friggeri-cv} % Add 'print' as an option into the square bracket to remove colors from this template for printing

\addbibresource{bibliography.bib} % Specify the bibliography file to include publications

\begin{document}

\header{marcel}{cutts}{full-stack engineer} % Your name and current job title/field

%----------------------------------------------------------------------------------------
%	SIDEBAR SECTION
%----------------------------------------------------------------------------------------

\begin{aside} % In the aside, each new line forces a line break
\section{contact}
Address withheld on public CV
~
Number withheld on public CV
~
\href{mailto:me@marcelcutts.com}{me@marcelcutts.com}
\href{http://www.marcelcutts.com}{marcelcutts.com}
\section{languages}
English fluency
German basic
\section{programming}
Python, JavaScript
C\#, Java, C++
SQL, PLSQL
CSS3 \& HTML5
\section{methodologies}
TDD, BDD
Scrum/Kanban
UX, UCD
\end{aside}


\section{synopsis}

Full stack developer blended with client facing consultancy experience.

From stacking together a PC salvaged parts to popular web applications, over the years I have absorbed technologies and systems making me a capable generalist. My thirst for learning endures, and I will continue to take skills and knowledge hostage into the future. 

When not sat behind a computer, I mostly climb rocks, fight heavy people, trek across the arctic and play guitar talentlessly.
%----------------------------------------------------------------------------------------
%	EDUCATION SECTION
%----------------------------------------------------------------------------------------

\section{education}

\begin{entrylist}
%------------------------------------------------
\entry
{2007--2012}
{Physics {\normalfont with industrial experience}}
{University of Bristol}
{\emph{Thesis on the superstate of b-meson oscillations.} }
%------------------------------------------------
\end{entrylist}

%----------------------------------------------------------------------------------------
%	WORK EXPERIENCE SECTION
%----------------------------------------------------------------------------------------

\section{employment}

\begin{entrylist}
%------------------------------------------------
\entry
{2011-- 201\_ }
{Tessella}
{Oxford, United Kingdon}
{\emph{Analyst Developer} \\
During my time at Tessella I have had opportunities to work on an exceptional variety of development and consultancy projects across domains. This has ranged from designing and implementing large scale n-tier applications to analysing and advising on processes in the consumer industry and client negotiations.\\
Internally, I have championed new technologies and methodologies, as well as led small teams and mentoring developers.} 
%------------------------------------------------
\entry
{2010}
{Deutsches Elektronen-Synchrotron}
{Hamburg, Germany}
{\emph{Particle Physicist} \\
My efforts at the DESY  institute focused on designing and implementing additions to
the proposed International Linear Accelerator software framework, which are intended to graphically review
and manipulate high volumes of complex data. In addition, I regularly gave presentations to international audiences to promote scientific updates from the department.} 
%------------------------------------------------
\entry
{2009--2010}
{De La Rue}
{London, United Kingdom}
{\emph{Research and Development Scientist} \\
De La Rue allowed me contribute to upcoming security
products across many areas of science. Throughout I developed my own ideas, including a python based image analysis suite that I pioneered, creating a quality assurance process and provided previously unavailable abilities to the research division. \\
In addition, I participated in desiging, prototyping and refining oncoming products, including  work in holographics, polymer development, photonic crystals and printed electronics.}
%------------------------------------------------
\entry
{2004-2010}
{Various part time or short term employment}
{Worldwide}
{\emph{Various service positions} \\
Number of roles during younger years ranging from park keeper to bar tender.}
%------------------------------------------------
\end{entrylist}

%----------------------------------------------------------------------------------------
%	RELEVANT EXPERIENCE SKILLS SECTION
%----------------------------------------------------------------------------------------

\newpage
\section{selection of relevant experience}
This section would normally contain a selection of experience relevant to a particular position, but as this edition will be public available on the web, I have included a varied assortment of projects below instead. \\
\\
\begin{entrylist}
%------------------------------------------------
\projectentry
{Data sandbox -  visualisation and analytical engines}
{\emph{JavaScript, Python, D3, HTML5/CSS3, Bottle, Jasmine, SqlLite, UX} \\
An academic institute is currently running a birth cohort study, in which the goal is to track a hundred thousand women and families through stages of pregnancy and the first year of the child’s life. This will provide novel scientific data detailing  child development throughout different populations throughout the UK.
A number of issues needed attention, however primarily there was no quantitative system for predicting recruitment of participants and the costs outcome of potential cohort study plans. Due to the number of stakeholders, there are a large number of possible parameters to consider, who's impact needs to be known before changes can be made.

To accommodate this, I developed an interactive visualisation. This lets the users play with the data and aspects of the plan, and see real-time updates to participant and cost metrics in an intuitive way. This is achieved through a rich JS client on the front end and a performant python bottle server handling computational and modelling aspects. As not all users would be academically inclined and stakeholders would share an interest, a fierce amount of attention was kept on the design and user experience.

Consequently, this gave the institute the power to not only quickly, but confidently make decisions in a time constrained field. Additionally, due to the design efforts, it found a secondary purpose as presentation tool for displaying figures and scenarios externally.}
\end{entrylist}

%------------------------------------------------
\begin{entrylist}
\projectentry
{Pharmaceutical portal}
{\emph{C\#, MSSQL, Entity Framework, MVC4, nUnit, nSubstitute, WF, WCF, Unity, CI} \\
The portal offers a number of services providing speciality pharmaceuticals, including a Global Access Program which is intended to provide unlicensed or trial-phase drugs to patients in need, internationally and domestically. A solution was required to handle requesting, validation, procurement shipping and tracking of these drugs by health care professionals, via pharmaceutical suppliers, while fulfilling all necessary industry regulations. As the software gained clients, complex analytics and reporting were developed to ain in overseeing the complex pharmaceutical chain.

A web-based application was created, by myself in a small scrum team, to satisfy these requirements. The resulted is a highly accessible portal for users and clients, which automated several previously manual processes. This allowed the client to rapidly expand its program range with minimal additional effort.}
\end{entrylist}
%------------------------------------------------

\begin{entrylist}
\projectentry
{Domain name and SSL}
{\emph{Java, WCF, Selenium, jUnit, nUnit, Oracle, PLSQL } \\
This project consisted of two core tasks: the creation of a full-stack web application to allow users to manage their domains independently and inclusion of new Domain Name System Security Extensions (DNSSEC) to secure domains against more modern forms of malicious network attacks.

The work involved alterations at the database level, writing packages to handle data directly into large currently existing Oracle systems  and front end user design. Alongside the application, a background web service was developed with WCF to allow other departments within the organisation to use the functionality of the application without having to write it themselves. All tasks were completed using continuous integration featuring both unit and front end testing.

The application has proven popular, handling all certificate and domain name issues for the organisation, increasing their product range and allowing the organisation to free up a large number of resources for other critical projects.}
\end{entrylist}


\begin{entrylist}
\projectentry
{County force intelligence and tasking system}
{\emph{C\#, Sharepoint, nUnit, MS SQL, networking} \\
A number of country police forces use a central system to brief and organise their resources and tasks. A number of new features and tools were requested to be implemented in the current system, based on Sharepoint 2007 platform.

The current system had been built and expanded in an ad-hoc impromptu manner; a well architected re-implementation was suggested, and while SharePoint was not the best possible platform for their needs, external forces existed ensuring its use. I created a modular decoupled architecture to best serve their current needs and offer clean separation and extensibility in the future. Alongside the core technical tasks, I introduced and automated deployment methods, advised on hardware issues, and trained the clients on future maintenance and use of the delivered system.}
\end{entrylist}







%----------------------------------------------------------------------------------------

\end{document}