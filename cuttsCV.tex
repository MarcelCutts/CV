%%%%%%%%%%%%%%%%%%%%%%%%%%%%%%%%%%%%%%%%%
% Friggeri Resume/CV
% XeLaTeX Template
% Version 1.0 (5/5/13)
%
% This template has been downloaded from:
% http://www.LaTeXTemplates.com
%
% Original author:
% Adrien Friggeri (adrien@friggeri.net)
% https://github.com/afriggeri/CV
%
% License:
% CC BY-NC-SA 3.0 (http://creativecommons.org/licenses/by-nc-sa/3.0/)
%
% Important notes:
% This template needs to be compiled with XeLaTeX and the bibliography, if used,
% needs to be compiled with biber rather than bibtex.
%
%%%%%%%%%%%%%%%%%%%%%%%%%%%%%%%%%%%%%%%%%

\documentclass[]{friggeri-cv} % Add 'print' as an option into the square bracket to remove colors from this template for printing
\usepackage{longtable}

\begin{document}

\header{marcel}{cutts}{full-stack engineer} % Your name and current job title/field

%----------------------------------------------------------------------------------------
%	SIDEBAR SECTION
%----------------------------------------------------------------------------------------

\begin{aside} % In the aside, each new line forces a line break
\section{contact}
Address withheld on public CV
~
Number withheld on public CV
~
\href{mailto:me@marcelcutts.com}{me@marcelcutts.com}
\href{http://www.marcelcutts.com}{marcelcutts.com}
\section{languages}
English fluency
German basic
\section{programming}
Python, JavaScript
C\#, Java, C++
SQL, PLSQL
CSS3 \& HTML5
\section{methodologies}
TDD, BDD
Scrum/Kanban
UX, UCD
\end{aside}


\section{synopsis}

Full stack developer blended with client facing consultancy experience.

From stacking together a PC salvaged parts to popular web applications, over the years I have absorbed technologies and systems making me a capable generalist. My thirst for learning endures, and I will continue to take skills and knowledge hostage into the future. 

When not sat behind a computer for work or for fun, I mostly climb rocks, fight heavy people, trek across the arctic and play guitar talentlessly.
%----------------------------------------------------------------------------------------
%	EDUCATION SECTION
%----------------------------------------------------------------------------------------

\section{education}

\begin{entrylist}
%------------------------------------------------
\entry
{2007--2012}
{Physics {\normalfont with industrial experience}}
{University of Bristol}
{\emph{Thesis on the superstate of b-meson oscillations.} }
%------------------------------------------------
\end{entrylist}

%----------------------------------------------------------------------------------------
%	WORK EXPERIENCE SECTION
%----------------------------------------------------------------------------------------

\section{employment}

\begin{entrylist}
%------------------------------------------------
\entry
{2011-- 201\_ }
{\LARGE{Tessella}}
{Oxford, United Kingdom}
{\emph{Analyst Developer} \\
During my time at Tessella I have had opportunities to work on an exceptional variety of development and consultancy projects across domains. This has ranged from designing and implementing large scale n-tier applications to analysing and advising on processes in the consumer industry.\\
Internally, I have championed new technologies and methodologies, as well as led small teams and mentoring developers. \\
\centerline{\textbf{Selected sample of completed projects}}}

\entry
{2013-2014}
{\textbf{Data visualisation - National cohort study}}
{}
{Tech: \emph{JavaScript, Python, D3, HTML5/CSS3, Bottle, Jasmine, SqlLite, UX} \\
Role: \emph{Consultant, sole developer} \\
An academic institute is currently running a national birth cohort study investigating 100,000 people over a number of years. My role as an analyst and developer was to provide quantitative feedback which outlined the possible impact of study steering decisions. \\
Once I had begun to understand the domain and the needs of the various stakeholders, I created and refined software that allowed the institute to visually play with their data and parameters in real-time, receiving instant feedback on their potential decisions through detailed predictive metrics.\\
I created this system through a strong JavaScript front-end utilising D3 and other libraries, and python for numerical work on the back-end. This was achieved using a number of agile methodologies which I later expanded the use of within the study as a whole to help the institute adapt more rapidly to changing requirements. \\
Consequently, this gave the institute the power to not only quickly, but confidently make decisions in a time constrained field. As an additional bonus, due to the design efforts, it found a secondary purpose as presentation tool for displaying figures and scenarios externally.

}

\entry
{2013}
{\textbf{Pharmaceutical portal - Specialist pharmaceutical supplier}}
{}
{Tech:  \emph{C\#, MSSQL, Entity Framework, MVC4, nUnit, nSubstitute, WF, WCF, Unity} \\ Role: \emph{Team developer. Later, tech lead.}\\
The portal offers a number of services providing speciality pharmaceuticals, including
a Global Access Program which is intended to provide unlicensed or trial-phase drugs
to patients in need, internationally and domestically.\\
A solution was required to handle
requesting, validation, procurement shipping and tracking of these drugs by health care
professionals, via pharmaceutical suppliers, while fulfilling all necessary industry regulations. As the software gained clients, analytics and reporting were developed
to aid in overseeing the complex pharmaceutical chain. \\
A web-based application was created, by myself in a small scrum team, to satisfy these
requirements. The resulted is a highly accessible portal for users and clients, which automated several previously manual processes. This allowed the client to rapidly expand
its program range with minimal additional effort. 
}

\entry
{2011}
{\textbf{Domain and security application - Large domain and network provider}}
{}
{Tech: \emph{Java, WCF, Selenium, jUnit, nUnit, Oracle, PLSQL} \hfill Role: \emph{Sole developer} \\
This project consisted of two core tasks: the creation of a full-stack web application to allow users to manage their domains independently and inclusion of new Domain Name System Security Extensions (DNSSEC) to secure domains against more modern forms of malicious network attacks.

The work involved alterations at the database level, writing packages to handle data directly into large currently existing Oracle systems  and front end user design. Alongside the application, a background web service was developed with WCF to allow other departments within the organisation to use the functionality of the application without having to write it themselves. All tasks were completed using continuous integration featuring both unit and front end testing.

The application has proven popular, handling all certificate and domain name issues for the organisation, increasing their product range and allowing the organisation to free up a large number of resources for other critical projects.
}

\end{entrylist}
\begin{asidenoheader} % In the aside, each new line forces a line break
\section{contact}
Address withheld on public CV
~
Number withheld on public CV
~
\href{mailto:me@marcelcutts.com}{me@marcelcutts.com}
\href{http://www.marcelcutts.com}{marcelcutts.com}
\section{languages}
English fluency
German basic
\section{programming}
Python, JavaScript
C\#, Java, C++
SQL, PLSQL
CSS3 \& HTML5
\section{methodologies}
TDD, BDD
Scrum/Kanban
UX, UCD
\end{asidenoheader}

\begin{entrylist}
%------------------------------------------------
\entry
{2010}
{\LARGE{Deutsches Elektronen-Synchrotron}}
{Hamburg, Germany}
{\emph{Particle Physicist} \\
My efforts at the DESY  institute focused on combining knowledge of physics and programming ability to contribute to an upcoming particle accelerator, the International Linear Collider. As part of my role, I regularly gave presentations to international audiences to promote scientific updates from the department and attended numerous specialised lectures. \\
\centerline{\textbf{Selected sample of completed projects}}} 

\entry
{2010}
{\textbf{Sensor data manipulation and review}}
{}
{Tech: \emph{C++, Qt, CMake, Linux} \hfill Role: \emph{Imagineer} \\
Particle accelerators hold an array of sensors to detect various energies and particles present, numbering in the thousands. The state of the detectors has significant impact on the interpretation of any possible results, and therefore require management.\\
I had sole responsibility of creating a system to visualise, review, manage these sensors which involved both an understanding of the physics at hand, as well as the software knowledge to integrate with the broader software suite being developed. \\
The project was completed and presented to an international audience. It is now primed to be used when the accelerator goes live in the future.
}

%------------------------------------------------
\\
\entry
{2009--2010}
{\LARGE{De La Rue}}
{London, United Kingdom}
{\emph{Research and Development Scientist} \\
De La Rue allowed me contribute to upcoming security
products across many areas of science. Throughout I developed my own ideas, mostly within the realm of software, but also within materials science and optics. \\
Within the broader team, I participated in designing, prototyping and refining oncoming products, including  work in holographics, polymer development, photonic crystals and printed electronics. \\
\centerline{\textbf{Selected sample of completed projects}}}

\entry
{2010}
{\textbf{Image analysis suite}}
{}
{\emph{Python, PIL} \\
During my time at the department, a significant shift was going through the security printing industry which focused less on chemistry in inks and watermarks and looked more towards advanced optical effects as deterrents  to counterfeiting. \\
Due to the traditional nature of the company, no programming and few quantitative data systems or processes existed. I proposed and spear-headed an image analysis suite that could output several metrics on the goodness of produced products, as well as checking for the viability of various new technologies. Technologically, this was achieved through a combination of python programming and hardware hacking with available supplies. \\
This created a brand new automated quality assurance process, which allowed products relying on tighter tolerances to be manufactured, and gave the research division previously unavailable insights.
}

\end{entrylist}

\begin{asidenoheader} % In the aside, each new line forces a line break
\section{contact}
Address withheld on public CV
~
Number withheld on public CV
~
\href{mailto:me@marcelcutts.com}{me@marcelcutts.com}
\href{http://www.marcelcutts.com}{marcelcutts.com}
\section{languages}
English fluency
German basic
\section{programming}
Python, JavaScript
C\#, Java, C++
SQL, PLSQL
CSS3 \& HTML5
\section{methodologies}
TDD, BDD
Scrum/Kanban
UX, UCD
\end{asidenoheader}








%----------------------------------------------------------------------------------------

\end{document}